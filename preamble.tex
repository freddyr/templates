% a preamble with al u need

% Dokumenteigenschaften bzw. Dokeumentenklasse
\documentclass[a4paper,
               10pt,
               fleqn]{article}
                         
               
% Texteigenschaften
\usepackage[utf8]{inputenc} % utf8x kann alle
                                % Textcodierungen
                                % interpretieren
\usepackage[T1]{fontenc} % Schriftcodierung mit UTF-8
\usepackage{textcomp} % Erweiterung von fontenc
\usepackage{lmodern} % Erweiterung des

\usepackage{courier}

\usepackage{dtklogos}

%\PrerenderUnicode{ä}
%\PrerenderUnicode{ü}
%\PrerenderUnicode{ö}


% Grafikpakete
\usepackage{graphics}
\usepackage{graphicx}

% Spracheigenschaften
\usepackage[ngerman]{babel} % ngerman = Neues
                                % Deutsch; babel =
                                % internationalisierung
                                % einschalten

% Links im PDF erzeugen (für Verzeichnisse, URLs etc.)
\usepackage{hyperref}

% Mathepakete
\usepackage{amsmath}
\usepackage[all]{xy}

% Glossarpakete
\usepackage[xindy]{glossaries}
\usepackage{makeidx}

% PDF-Pakete
\usepackage{pdfpages}

% Spezielle Grafikpakete
\usepackage{graphicx}

% Source-Code Paket
\definecolor{darkgreen}{rgb}{0,0.6,0}
\usepackage{listings}
\lstset{language=[LaTeX]TeX}
\lstloadlanguages{TeX}
\lstset{basicstyle=\ttfamily,
        numbers=left,
        numberstyle=\tiny,
        numbersep=5pt,
        breaklines=true,
        texcsstyle=\color{black},
        backgroundcolor=\color{gray!10},
        commentstyle=\color{darkgreen},
        %keywordstyle=\color{red}\bfseries,
        %stringstyle=\color{blue}\bfseries,
        frame=single,
        tabsize=2,
        rulecolor=\color{black!30},
        title=\lstname,
        escapeinside={\%*}{*)},
        breaklines=true,
        breakatwhitespace=true,
        framextopmargin=2pt,
        framexbottommargin=2pt,
        inputencoding=utf8,
        extendedchars=true,
        literate={Ö}{{\"O}}1
                 {Ä}{{\"A}}1
                 {Ü}{{\"U}}1
                 {ü}{{\"u}}1
                 {ä}{{\"a}}1
                 {ö}{{\"o}}1 }
    
% 
\usepackage{printlen}

% Lorem-Ipsum Paket
\usepackage{blindtext}  % generiert sprachlich korrekten "Fülltext"
\usepackage{lipsum} % generiert klassischen "lorem-ipsum"

% Euro-Betrag Zeichen Paket
\usepackage{eurosym}

% Abkürzungs-Paket
\usepackage{acronym}

% Auflistungs-Paket (für Auflistungen mit "a)", "b)"...
\usepackage{enumitem}

% URL-Paket (URLs richtig darstellen und umbrechen z.B. in Literaturverzeichnissen
\usepackage{url}

% Zitier-Paket
\usepackage{cite}   % allgemeines Paket
\usepackage{apacite}    % Zitat-Paket für APA-Norm Zitate

% Kopf- und Fusszeilen Paket
\usepackage{fancyhdr}

% Spezial Titelseite
%\usepackage[affil-it]{authblk}

%%%%%%%%%%%%%%%%%%%%%%%%%%%%%%%%%%%%%%%%%%%%%%%%%%%%%%%%%%%%%%%%%%%%%%%%%%%%%%%%
%%% Kopf und Fusszeilen definieren
%%%%%%%%%%%%%%%%%%%%%%%%%%%%%%%%%%%%%%%%%%%%%%%%%%%%%%%%%%%%%%%%%%%%%%%%%%%%%%%%

\pagestyle{fancy}   % deklaieren dass ein eigener Syle benutzt wird, eben "fancy"
\fancyhf{}  % alle Kopf- und Fusszeilenfelder bereinigen

\addtolength{\textwidth}{1cm}   % anpassen der textbreite
\addtolength{\evensidemargin}{-5mm} % anpassen des Einzugs für gerade Seiten
\addtolength{\oddsidemargin}{-5mm}  % anpassen des
                                       % Einzugs für
                                       % ungerade Seiten

\renewcommand{\sectionmark}[1]{\markright{#1}{}}


\addtolength{\headwidth}{1cm}   % anpassen der Kopf/Fusszeilenbreite (Summe von den Oberen)

\fancyhead[L]{} % Kopfzeile links
\fancyhead[C]{} % Kopfzeile mitte
\fancyhead[R]{Verein zur Förderung der ICT Berufsbildung}   % Kopfzeile rechts

\renewcommand{\headrulewidth}{0.4pt} % obere Trennlinie

\fancyfoot[L]{Überbetrieblicher Kurs 304}   % Fusszeile links
\fancyfoot[C]{\today}   % Fusszeile mitte
\fancyfoot[R]{Seite \thepage} % Fusszeile rechts

\renewcommand{\footrulewidth}{0.4pt} %untere Trennlinie

%%%%%%%%%%%%%%%%%%%%%%%%%%%%%%%%%%%%%%%%%%%%%%%%%%%%%%%%%%%%%%%%%%%%%%%%%%%%%%%%
%%% Ende der Präambel
%%%%%%%%%%%%%%%%%%%%%%%%%%%%%%%%%%%%%%%%%%%%%%%%%%%%%%%%%%%%%%%%%%%%%%%%%%%%%%%%





% Version
\newboolean{kursleiter} %Deklaration
\setboolean{kursleiter}{true} %Zuweisung

%Definition vom ``Beachten sie''
\newenvironment {beachte}
                [0]
                {   \color{blue} \noindent
                    \textbf{Beachten sie:} \newline \noindent 
                }
                { \color{black} }

% Definition vom ``Zwingend''                
\newenvironment {muss}
                [0]
                {   \color{red} \noindent
                    \textbf{Zwingend:} \newline \noindent 
                }
                { \color{black} }
% Definition von ``betone''
\newcommand{\betone}[1]{\textbf{#1}}


% Schriftart ``word-like-Arial''
%\usepackage{arev}       % um den Default wieder herzustellen einfach diese 
                        % Zeile mit einem Prozentzeichen auskommentieren
                        % Default = Computer Modern
                        % siehe   http://en.wikibooks.org/wiki/LaTeX/Fonts
                        % und     http://www.tug.dk/FontCatalogue/